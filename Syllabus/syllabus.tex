\documentclass[11pt,twocolumn]{article}
\usepackage[mathletters]{ucs}
\usepackage[utf8x]{inputenc}
\usepackage[text={6in,9.5in},centering]{geometry}

\usepackage{ifpdf}
\usepackage{graphicx}
\usepackage{sectsty}
\allsectionsfont{\sffamily}
\setlength{\parindent}{0pt}
%\setlength{\parskip}{6pt plus 2pt minus 1pt}


\usepackage[breaklinks=true]{hyperref}
\title{Internet Programming}
\setcounter{secnumdepth}{0}
\author{}
\begin{document}

\begin{center}
	{\sffamily\LARGE\bfseries CS-330 Internet Programming}
\end{center}

\section*{Instructor}
Brad Miller \\
Olin, 321 \\
email:  bmiller@luther.edu \\
Skype:  bonelake \\
Google+ \\

\section*{Office Hours}
Monday (2-4) Wednesday, Friday: 1--2:30 \\
Other times by appointment, drop-in, or virtual.  Really! I'm here to help you, so stop in.

\section*{Text Book}

I have never found a good textbook that covers the materials we will cover in this course.  I will continue to update the chapters I have been working on in the online webfundamentals book.

\section*{Goals}

\begin{itemize}
    \item To understand the common technologies used in building web applications today
    \item To understand how modern web publishing works
    \item To learn how to learn about rapidly changing technologies
    \item To become fluent in the Javascript Language
    \item To understand the mechanisms behind modern web frameworks
\end{itemize}

\section*{Course Outline}
\begin{enumerate}
	\addtolength{\itemsep}{-0.5\baselineskip}
    \item Server Side Programming
	\begin{enumerate}
		\item Generating dynamic web pages  (Old Skool CGI)
		\item Processing forms
		\item GET versus POST
		\item Baking cookies
	\end{enumerate}
	\item Javascript
	\begin{enumerate}
 		\item Language fundamentals Short Review
		\item The Document Object Model (DOM)
		\item Closures
		\item Prototypes
		\item AJAX
		\item JSON
		\item jQuery, AngularJS and other JS frameworks
        \item Bootstrap -- Responsive pages for (almost) free
	\end{enumerate}
	\item Midterm Mashups
	\item From Webserver to Application Server
	\begin{enumerate}
		\item A 10 line webserver in Python
		\item From static to dynamic
		\item understanding WSGIref
		\item Template Systems
	\end{enumerate}
	\item Application Servers
	\begin{enumerate}
		\item Using the Flask Framework
		\item Using a database
		\item Node.js
	\end{enumerate}
	\item Final Project
\end{enumerate}

\section*{Class Requirements:}

\subsection*{Participation}

``The teacher's job is to design learning experiences, not primarily to impart information''  This is going to be an experiential class, I'll lecture some and try to explain some big picture stuff but, I've learned everything I know in this area by doing stuff.  I expect you to be engaged and to ask questions and do stuff.  If you stop at the bare minimum of what I ask you to do for any particular assignment you'll be missing an opportunity.

\subsection*{Homework}

You will have at least weekly homework assignments, sometimes more frequently when they are easier.

\subsection*{Midterm Mashups}

The mid-term mashup will be a substantial part of your grade.  You will need to design and implement a mashup, then you'll present a demo of it to the class when it is all done.  Many of today's successful web businesses started out as simple mashups.  If you desire you can work with one other person on the mid-term.
	\begin{itemize}
	    \item Create a web page/pages that utilize one or more Web APIs
	    \item Built with Javascript, CSS, and HTML
	    \item Examples of available APIs
	    \item Google Documents, Maps, Calendar, etc.
	    \item Twitter, Facebook
	    \item Amazon, Evernote, Dropbox, RTM, tweedledo
	    \item Movies, Music, Books, ...
	    \item 1000's of others
	    \item Combine one or more of these web services to create your own cool service
	\end{itemize}

\subsection{Final Project} % (fold)
\label{sub:final_project}
Later in the semester you will need to devise a final project.  This project will make use of the web development framework we've used in class, or you could venture out and use one of the many other web frameworks (Django, Ruby on Rails, Groovy on Grails, Google App Engine, etc.)  You can work in a group of up to 3.  You will need to write up a formal proposal for this project and have it approved by me before starting.  You and your group will present the project to the class in the last week of the course.

% subsection final_project (end)
\subsection*{Grading}


\begin{verbatim}
30%   Midterm Mashup
30%   Homework Assignments
30%   Final Project
10%   Class participation
\end{verbatim}

The grading scale is:

\begin{verbatim}
90 -- 100   A
80 -- 89    B
70 -- 79    C
60 -- 69    D
0 --  59    F
\end{verbatim}


\end{document}
